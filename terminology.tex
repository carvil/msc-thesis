\bthisterminology{10}
\termitem{Abstract Class}{
An abstract class is an \writeterm{abstract type} containing variables and methods (with or without bodies), whose purpose is to act as a place-holder for additional variables and methods to be added by subclasses.\\
It is not possible to instantiate abstract classes. Furthermore, it is not necessary to have a complete implementation of it, leaving the implementation details to a class that \writeterm{extends} the abstract class. Thus, it is possible for each subclass to override the defined methods in the abstract class specializing them according to their own needs.\\
Moreover, the abstract classes specify a public interface which can be inherited by its subclasses, allowing all the subclasses to have the same interface.\\
Finally, Java does not allow multiple inheritance concerning to abstract classes. A non-abstract class can only extend one abstract class, and an abstract class can be extended by a number of classes.
}
\termitem{Abstract Syntax Tree}{
Is a tree representation of the syntax of a given source code. After parse a given input file with a parser, an AST is created with the syntactic information of the input file.}

\termitem{Class}{
Unlike the abstract class, a class, composed by variables and methods, must provide a complete implementation of their own methods.\\ 
Furthermore, a class is meant to have objects instantiated from it, providing those objects has the same variables and methods, and letting them have their own values.\\
As it was explained above, a class can implement a number of Interfaces and only extend one class/abstract class. For each Interface implemented, the class must complete its methods and the same happens for the extended abstract class.\\
Furthermore, it is also possible to extend one non-abstract class, and re-implement all its methods.
}
\termitem{Interface}{
An Interface is conceptually an \writeterm{abstract type} composed simply by method headers and static final variables (constant declarations), whose purpose is to specify how the interconnection between different systems and the one in question should work.\\ 
Furthermore, the implementation of the methods and the creation of non-static final variables is forbidden in an Interface, and should be left for an implementation class.\\ 
Each class (including abstract classes) can implement a number of interfaces, and it should specify the methods presented in each one. In fact, this is the only kind of multiple inheritance allowed in Java.
}
\termitem{Object-Oriented paradigm}{
Is a programming paradigm that makes use of objects are a mean of interact with design applications and computer programs.
}
\termitem{Pure Type}{
Pure types are immutable types, side-effect free. This means that after the creation of an object of a pure type, its value cannot be changed in any state of the execution of a program.}

\ethisterminology
