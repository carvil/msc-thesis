\chapter{Input/Output Stream}
\label{appendix3}


In this appendix, the specification of the case study Input/Output Stream is presented. Each section contains the specification and the results of applying the mapper, of one class. Thus, section \ref{app3:io} contains the \vpp\ specification of the \textit{IO} class and the resulting \jml\ specification, after applying the mapper. The same happens for each of the other classes.


\section{IO}
\label{app3:io}

\subsection{Specification}

\lstset{style=mystyle}
\bigskip
\begin{lstlisting}
class IO

types

public byte = nat;

operations 

public read : seq of byte  ==> nat
read(b) == is subclass responsibility;

public write : seq of byte  ==> ()
write(b) == is subclass responsibility;

public close : () ==> ()
close() == is subclass responsibility;

end IO
\end{lstlisting}
\bigskip


\section{InputStream}

\subsection{Specification}

\lstset{style=mystyle}
\bigskip
\begin{lstlisting}
class InputStream is subclass of IO

operations

public read : seq of byte ==> nat
read(b) == is not yet specified;

public close : () ==> ()
close() == is not yet specified;


end InputStream
\end{lstlisting}
\bigskip


\section{OutputStream}

\subsection{Specification}

\lstset{style=mystyle}
\bigskip
\begin{lstlisting}
class OutputStream is subclass of IO

operations

public write : seq of byte ==> ()
write(b) == is not yet specified;

public close : () ==> ()
close() == is not yet specified;

end OutputStream
\end{lstlisting}
\bigskip


\section{IOStream}

\subsection{Specification}

\lstset{style=mystyle}
\bigskip
\begin{lstlisting}
class IOStream is subclass of InputStream, OutputStream

operations

public read : seq of byte ==> nat
read(b) == is not yet specified;

public write : seq of byte  ==> ()
write(b) == is not yet specified;

public close : () ==> ()
close() == is not yet specified;

end IOStream
\end{lstlisting}
\bigskip

