% ********** Appendix 1 **********
\chapter{Specification of the JML Abstract Representation}
\label{appendix:jmlast}

In this appendix, it is possible to see the complete specification of the \jml\ abstract syntax. The appendix is divided in several sections, each one containing a complete block of the specification. Furthermore, each type is commented in order to understand what it represents. This AST is then converted, using the \textit{ASTGEN} (see \ref{chap4:sec:asts}) into \java\ and \vpp\ classes in order to be used within the specification. 

\section{\jml\ Specifications}

The abstract representation of \jml\ specifications assumes the possibility of existing more than one file, which means more than one class, or even nested classes. Thus, a \jml\ specifications can be represented as a sequence of \jml\ classes, as represented below.
\lstset{style=AST}
\lstset{language=VDM++}
\medskip
\begin{lstlisting}
Specifications ::
	class_list : seq of WrappedJmlClass
	;
\end{lstlisting}
\medskip
Each \jml\ class is composed by a package name, a refinement clause, the correspond type imports and by the class body itself.
\medskip
\begin{lstlisting}
WrappedJmlClass ::
	package : seq of char
	refine : seq of char
	imports_java : seq of Import
	imports_jml : seq of ModelImport
	class_val : Class
	;
\end{lstlisting}
\medskip
Each import is nothing more than a string. However, it is possible to have \java\ imports, for \java\ types, and \jml\ imports, for the \jml\ pure types.
\medskip
\begin{lstlisting}
Import ::
	import : seq of char
	;

ModelImport ::
	model : bool
	import : seq of char
	;
\end{lstlisting}
\medskip
Concerning the class, a \jml\ class can be represented by the \vdm\ type presented below. A class is composed by an access definition, which represents the class visibility, a keyword representing the class kind (interface, class, etc), the class name, both the interface and class inheritance clause and the sequence of definition blocks (operations, etc).
\medskip
\begin{lstlisting}
Class ::
  access : AccessDefinition
  kind : ClassKind
  identifier : Identifier
  inheritance_clause : [ClassInheritanceClause]
  interface_inheritance : [InterfaceInheritanceClause]
  class_body : seq of DefinitionBlock
  ;
\end{lstlisting}
\medskip
A few types related to the class.
\medskip
\begin{lstlisting}
ClassKind =
	<CLASS> | <ABSTRACT> | <INTERFACE>
	;

InterfaceInheritanceClause ::
	identifier_list : seq of Identifier
	;

ClassInheritanceClause ::
	identifier_list : Identifier
	;

AccessDefinition :: 
	scope  : Scope
	;
  
Scope =
  <PUBLIC> | <PRIVATE> | <PROTECTED> | <DEFAULT>
  ;
\end{lstlisting}
\medskip
A definition block can be an instance variable block, a value block (static variables), an invariant block, an operation block or other definitions, which represents the constructs at the \vpp\ side that do not have an association at the \jml\ side, and vice-versa.
\medskip
\begin{lstlisting}

DefinitionBlock =
	InstanceVariableDefinitions |
	ValueDefinitions |
	InvariantDefinitions | 
	OperationDefinitions |
	OtherDefinitions
	;

OtherDefinitions :: ;

\end{lstlisting}
\medskip

\section{Instance Variables}
In this section, the instance variables block is presented. It is possible to have both \jml\ and \java\ variables inside a specification.
\medskip
\begin{lstlisting}
InstanceVariableDefinitions :: 
	jml_variables : seq of Variable
	java_variables : seq of Variable
	;
\end{lstlisting}
\medskip
Furthermore, each variable has a number of information fields, represented by the type below.
\medskip
\begin{lstlisting}
Variable ::
	access : AccessDefinition
	model : bool
	statickeyword : bool
	final : bool
	type : Type
	identifier : Identifier
	expression : [Expression]
	;
\end{lstlisting}

\section{Values}

A value in \jml\ a static variable, which corresponds to the \vpp\ value. Below, it is possible to see the abstract definition of values in \jml.

A value definition block is a sequence of values.
\medskip
\begin{lstlisting}
ValueDefinitions ::
    value_list : seq of ValueDefinition
	;
\end{lstlisting}
\medskip
Each value has the following definition:
\medskip
\begin{lstlisting}
ValueDefinition ::
	access : AccessDefinition
	static_mod : bool
	final_mod : bool
	shape: ValueShape
	;
	
ValueShape ::
	identifier : Identifier
	type : Type
	expression : Expression
	;
\end{lstlisting}

\section{Invariants}

The following definition represents invariants in \jml. It is possible to have a sequence of invariants, each one with a particular definition shown below.
\medskip
\begin{lstlisting}
InvariantDefinitions :: 
	invariant_list : seq of InvariantDefinition;

InvariantDefinition ::
	access : AccessDefinition
	redundant : bool
	predicate : Expression
	;
\end{lstlisting}

\section{Operations}

The operations block is composed by a number of operations, represented by a sequence.
\medskip
\begin{lstlisting}
OperationDefinitions ::
	operation_list : seq of OperationDefinition
	;
\end{lstlisting}
\medskip

Each operation is represented by a number of fields, already explained in chapter \ref{chapter4}.

\medskip
\begin{lstlisting}	
OperationDefinition :: 
	trailer : [MethodSpecifications]
	access : AccessDefinition
	pure : bool
	statickeyword : bool
	final : bool
	returning_type : Type
	identifier : Identifier
	parameter_list : seq of Parameter
	body : [Body]
	;
\end{lstlisting}
\medskip

AN operation can have new specifications or an extension to other specifications (through inheritance). There are three kinds of specifications: normal, behaviour and exceptional specifications.

\medskip
\begin{lstlisting}
MethodSpecifications ::
	specs : Specs
	also  : Specs
	;
	
Specs ::
	list : seq of OperationTrailer
	;

OperationTrailer =
	BehaviourSpec |
	ExceptionalSpec |
	NormalSpec
	;
\end{lstlisting}
\medskip
Each kind of specifications has a scope and a number of trailers (for example, pre-conditions).
\medskip
\begin{lstlisting}
BehaviourSpec ::
	privacy : Scope
	list : seq of Trailers
	;

ExceptionalSpec :: 
	privacy : Scope
	list : seq of Trailers
	;
	
NormalSpec :: 
	privacy : Scope
	list : seq of Trailers
	;
\end{lstlisting}
\medskip
There are five possible trailers allowed by \jml, listed below.
\medskip
\begin{lstlisting}	
Trailers =
	EnsuresClause |
	AssignableClause | 
	SignalsClause |
	SignalsOnlyClause | 
	RequiresClause
	;
\end{lstlisting}
\medskip
Each of the five possible trailers are described below. The common field between them is an expression, which will be explained further on.
\medskip
\begin{lstlisting}
EnsuresClause ::
	ensures_expression : Expression
	;
	
RequiresClause ::
	requires_expression : Expression
	;
	
AssignableClause ::
	assignable_list : seq of Identifier
	;
	
SignalsClause ::
	exception : Exception
	predicate : Expression
	;

SignalsOnlyClause ::
	reference_types : seq of ReferenceType
	;
\end{lstlisting}
\medskip
Below, one can see some useful types used above, and besides that, the representation of the body (as string) because this connections does not consider the connection between implementation of operations.
\medskip
\begin{lstlisting}
ReferenceType = 
	Identifier 
	;

Exception ::
	type : ExceptionType
	identifier : Identifier
	;

Body :: 
	body : seq of char
	;

Parameter :: 
	type : Type
	identifier : Identifier
	;
\end{lstlisting}


\section{Types}

In this section, it is possible to see the representation of some of the types chosen to be used at the \jml\ side. Below, one can see those types.

\begin{lstlisting}
Type =
	CharType |
	BoolType |
	EnumerationType | 
	IntegerType |
	FloatType |
	MapValueToValueType | 
	SetValueType |
	SeqValueType |
	ObjectType |
	VoidType |
	TupleType |
	NatType |
	ClassType |
	ClassName
	;
\end{lstlisting}

This first set of types represent \jml\ classes, which will be mapped from \vpp\ composite types.

\begin{lstlisting}
ClassName ::
	id : Name;

ClassType ::
	id : Identifier
	field_list : seq of Field;

Field ::
	id : Identifier
	type : Type;
\end{lstlisting}

This set of types represent \jml\ types which will be mapped from \vpp\ simple types.

\begin{lstlisting}
NatType :: 
	limit : nat;

TupleType ::
	vals : seq of Type;

CharType :: ;

BoolType :: ;

IntegerType :: ;

EnumerationType ::
	enum_literal : EnumLiteral;

FloatType :: ;
\end{lstlisting}

Finally, \vpp\ complex types can be mapped to the \jml\ types represented below.

\begin{lstlisting}
MapValueToValueType ::
	dom_type : Type
	rng_type : Type;

SetValueType ::
	type : Type;

SeqValueType ::
	type : Type
	limit : nat;

ObjectType :: ;

ExceptionType ::
    type : Identifier;

VoidType :: ;
\end{lstlisting}

\section{Expressions}

In this section, a number of \jml\ expressions are presented. This expressions are used to map from \vpp\ expressions, and are used in several parts of a specification, such as pre-conditions, invariants, etc.

\begin{lstlisting}
Expression = BracketedExpression |
             IfExpression |
			 UnaryExpression |
			 BinaryExpression |
			 ForAllExpression |
			 ExistsExpression |
			 OldName |
			 NewExpression |
			 Name |
			 PostfixExpression |
			 SetEnumeration |
			 SequenceEnumeration |
			 MapEnumeration |
			 ApplyExpression |
			 FieldSelectExpression |
			 ArrayExpression |
			 ThrowExpression |
			 LiteralExpression |
			 BlockExpression |
			 ThisExpression |
			 InstanceOfExpression |
			 OldName
			 ;
\end{lstlisting}
\medskip
The following types represent the \textit{old} expression, the \textit{instanceof} expression and the \textit{this} expression.
\medskip
\begin{lstlisting}
OldName ::
	identifier : Identifier;

InstanceOfExpression :: 
	type : Type
	expression : Expression;

ThisExpression :: ;
\end{lstlisting}
\medskip
Below, one can find types representing the \jml\ map enumeration, the sequence enumeration and the set enumeration.
\medskip
\begin{lstlisting}
MapEnumeration ::
	list : seq of MapLet;
	
MapLet ::
	dom_val : Expression
	rng_val : Expression;

SequenceEnumeration ::
	list : seq of Expression;

SetEnumeration ::
	list : seq of Expression;
\end{lstlisting}
\medskip
The block expression represents the \jml\ type from which the \vpp\ let expression maps to. It has bind information and an expression. Furthermore, one can see definitions for an array expression and a throw expression.
\medskip
\begin{lstlisting}
BlockExpression ::
	bind : seq of ValueShape
	return_expr : Expression;

ArrayExpression ::
	list : seq of Expression
	;
	
ThrowExpression ::
	exception : Identifier
	params : seq of Parameter
	;
\end{lstlisting}
\medskip
Below, it is possible to see definitions for the apply expression, and the field select expression.
\medskip
\begin{lstlisting}
ApplyExpression ::
    expression : Expression
	expression_list : seq of Expression
	;

FieldSelectExpression ::
    expression : Expression
	name : Name
	;
\end{lstlisting}
\medskip
The definitions present below represents the bracketed expression in \jml, and the if-then-else clause.
\medskip
\begin{lstlisting}
BracketedExpression ::
	expression : Expression
	;
	
IfExpression ::
	if_expression : Expression
	then_expression : Expression
	else_expression : Expression
	;
\end{lstlisting}
\medskip
The unary expression has an operator, defined further on, and an expression in which the operator will operate.
\medskip
\begin{lstlisting}	
UnaryExpression ::
	operator : UnaryOperator
	expression : Expression
	;
\end{lstlisting}
\medskip
There are a number of unary operators allowed in \jml. One can increment or decrement the value of some expression. Furthermore, it is possible to use the unary plus and minus operators. Besides that, it is possible to use the not, old, absolute value, floor, domain, range, inverse, cardinality, power, head, tail, length and elements. Each one of this operators will have to respect the type of the expression in which they are being applied. 
\medskip
\begin{lstlisting}
UnaryOperator =
	<INCREMENT> | <DECREMENT> | 
	<PLUS> | <MINUS> |
	<NOT> | <OLD> | <ABS> |
	<FLOOR> | <DOM> | <RNG> |
	<INVERSE> | <CARD> | <POWER> |
	<HD> | <TL> | <LEN> | <ELEMS>
	;
\end{lstlisting}
\medskip
The binary expression is similar to the unary expression, unless it has one more expression. The binary operators are explained further on.
\medskip
\begin{lstlisting}
BinaryExpression :: 
	lhs_expression : Expression
	operator : BinaryOperator
	rhs_expression : Expression
	;
\end{lstlisting}
\medskip
It is possible to chose from a number of operators. They are represented by the type below. 
\medskip
\begin{lstlisting}
BinaryOperator = 
	<MINUS> | <PLUS> | <MULTIPLY> |
	<DIVIDE> | <REMAIN> | <LE> |
	<L> | <G> | <GE> | <EQ> | <NE> |
	<INSTANCEOF> | <LAND> | <LOR> |
	<IMPLY> | <IMPLYBACK> | <EQUIV> |
	<NOTEQUIV> | <MULEQ> | <DIVEQ> | 
    <REMEQ> | <PLUSEQ> | <MINUSEQ> |
	<SUBTYPE> | <MUNION> | <DOMRESTTO> |
	<RNGRESTTO> | <COMP> | <INSET> |
	<NOTINSET> | <UNION> | <INTER> |
	<SUBSET> | <PROPERSUBSET> | <CONCAT>
	;
\end{lstlisting}
\medskip
Below, it is possible to see the definition of the quantification expressions \textit{forall} and \textit{exists}. They are similar in a way they both have quantifier declarations and an expression.
\medskip
\begin{lstlisting}
ForAllExpression ::
	bind_list : QuantifierDeclaration
	expression : seq of Expression
	;
	
ExistsExpression ::
	bind_list : QuantifierDeclaration
	expression : seq of Expression
	;
\end{lstlisting}
\medskip
Each quantifier declaration mentioned above can have a bound, a type and a sequence of variables.
\medskip
\begin{lstlisting}	
QuantifierDeclaration ::
	bound : [BoundModifiers]
	type : Type
	vars : seq of Identifier
	;
	
BoundModifiers =
	<NON_NULL> |
	<NULLABLE>;
	
OldName ::
	identifier : Identifier
	;
\end{lstlisting}
\medskip
From the definitions below, it is possible to see the definitions of the \textit{new} expression and the name expression, which can represent a method name, for example.
\medskip
\begin{lstlisting}	
NewExpression ::
	type : Type
	expression_list : seq of Expression
	;
	
Name ::
	class_identifier : [Identifier]
	identifier : Identifier
	;
\end{lstlisting}
\medskip
Below, a postfix expression is represented.
\medskip
\begin{lstlisting}
PostfixExpression :: 
	primary : seq of PrimaryExpressions
	operation : PostfixOperation	
	;

PostfixOperation =
	<INCREMENT> |
	<DECREMENT>
	;
\end{lstlisting}
\medskip
Furthermore, a \jml\ primary expression is defined below.
\medskip
\begin{lstlisting}
PrimaryExpressions ::
	primary : PrimaryExpression
	suffix : [PrimarySuffix]
	;
\end{lstlisting}
\medskip
Some \jml\ keywords are represented below.
\medskip
\begin{lstlisting}
SuperKeyword :: ;
ThisKeyword :: ;
NullKeyword :: ;
ResultKeyword :: ;
OldKeyword :: ;
NotAssignedKeyword :: ;
StarKeyword :: ;
ClassKeyword :: ;
\end{lstlisting}
\medskip
The primary suffix in \jml\ can be one of four different types, listed below.
\medskip
\begin{lstlisting}
PrimarySuffix =
	ThisKeyword |
	ClassKeyword |
	SuperKeyword |
	ExpressionsList
	;
\end{lstlisting}
\medskip
Moreover, one can see the definitions for the \jml\ primary expressions and the literal expression, responsible for the usage of literals such as natural numbers, etc.
\medskip
\begin{lstlisting}
ExpressionsList ::
    list : seq of Expression
	       ;

PrimaryExpression =
    PrimaryExpressionLiteral |
	PrimaryExpressionKeyword
	;

PrimaryExpressionLiteral ::
    lit : Literal
	;
	
LiteralExpression ::
	lit : Literal
	;
	
PrimaryExpressionKeyword ::
    keyword : PrimaryExpressionOption
	;
\end{lstlisting}
\medskip
A primary expression can be one of the following types.
\medskip
\begin{lstlisting}
PrimaryExpressionOption = 
	NameId |
	SuperKeyword |
	ThisKeyword | 
	NullKeyword |
	ResultKeyword |
	OldKeyword |
	NotAssignedKeyword
	;

NameId ::
    name : Identifier
	;
\end{lstlisting}
\medskip

\section{Literals}

In this section, the definition of \jml\ literals can be found. As one can see below, there are a number of possible literals to use in \jml\ specifications. 

\medskip
\begin{lstlisting}
Literal =
	NumericalLiteral |
	FloatLiteral |
	EnumLiteral |
	BooleanLiteral |
	CharacterLiteral |
	StringLiteral |
	NullLiteral;
\end{lstlisting}	
\medskip
Below, one can see definitions about enumeration literals, numerical and float literals.
\medskip
\begin{lstlisting}
EnumLiteral	::
	val : seq of char;
	
NumericalLiteral ::
	val : nat;
	
FloatLiteral ::
	val : real;
\end{lstlisting}
\medskip
Furthermore, the definition of character and boolean literals is presented.
\medskip
\begin{lstlisting}
BooleanLiteral ::
	val : bool;
	
CharacterLiteral ::
	val : char;
\end{lstlisting}	
\medskip
Finally, it is presented the definition of string and null literals and the definition of an identifier, which is a sequence of characters (string).
\medskip
\begin{lstlisting}
StringLiteral ::
	val : seq of char;
	
NullLiteral :: ;

Identifier = seq of char
\end{lstlisting}




% ********** End of appendix **********
